Here is the text of a conversation between a teacher and a student (somewhat edited), discussing an interesting problem:
Teacher: Suppose I have a curve $y = x^2$ and then I revolve this curve around the y-axis. What's the volume of the enclosed shape if I go from $y=0$ to $y=1$? Any idea how to do that?
Student: Maybe take the integral? From 0 to 1? So something like $\int^1_0 x^2 \, dx$, except I don't think that's quite right.
Teacher: Well, that gives you the area under a curve. But we're saying that we spin this curve around the y axis. This traces out a volume.
Student: There needs to be another variable in the integral, I think. How about something like $\int^{y=1}_{y=0} y-x^2 \, dxy$? That also seems incorrect, but perhaps closer.
Teacher: Well, let's do some reasoning before we write integrals. Do you know the area of a circle with radius $r$?
Student: $\pi r^2$, and circumference is $2\pi r$. 
Teacher: Yeah, okay. Prove it. With calculus. Any idea how to do that?
Student: Let me see...it involves an integral, I'm guessing. And then, maybe a definite one, over the interval $0$ to $\pi$? No, that's not right. Well, maybe. but what are you integrating over? Not $r$, if it is a definite integral. Let me think about this more intuitively.
you rotate the radius around $\pi r$ times, I suppose, to get the area (does that make sense?) and then it might be instead over the interval $\int^{\pi}_r$
Teacher: Let's think of the disk as a set of concentric rings.
Student: and then, hmm, inside, you'd have some variable $x$, and how would that look...
maybe $x^2$? Or something?
so then integrating that you get $x^3/3$
no wait, that doesn't make sense either, because then you are subtracting
okay, different way
$2\pi r$ is the circumference of the circle
so then we can say $2\pi r$ over the interval $0$ to $r$?
so then you'd have the integral $\int^r_0 2\pi r \, dr$
and then you'd get $\frac{2\pi r^2}{2}\mid^r_0$
Teacher: Okay, what does that become?
Student: well, $\frac{2\pi r^2}{2}-\frac{2\pi 0^2}{2}$
the second part all becomes zero
so now you have $\frac{2\pi r^2}{2}$
and then $\pi r^2$
Teacher: Now review in your mind how you got that answer. You imagine a bunch of rings, each of radius r and length $2 \pi r$.
If you give that ring a little bit of radial extent $dr$, the area of that new thing is $2 \pi r \times dr$.
Student: Radial extent is how thick the ring is, right?
Teacher: Yes. Then you summed (i.e. integrated) from 0 to $R$ and got $\pi R^2$.
Student: I wonder how you prove $2\pi r = \text{circumference}$
Teacher: do you know how to relate arc length to angle?
i.e., what is a radian?
Student: barely...
Teacher 2: Well, that's the point where it gets tricky - one often defines $2\pi$ as the ratio of a circle to its radius, so there's nothing to prove.
There are other definitions (involving more math) of $\pi$ in which $C = 2\pi r$ is an actual theorem but I have to admit that I don't actually know how to derive it from them
Teacher: I was hoping to prove that the circumference of a circle is equal to the radius times a constant.
Teacher 2: Ah, nevermind. Carry on.
Teacher: Suppose I move a little bit horizontally $dx$ and a little bit vertically $dy$.
How far did I move in total?
Student: well, maybe $\sqrt{dx + dy}$? I don't know if that's what you are looking for.
Teacher 2: Actually $ds=\sqrt{dx^2+dy^2}$.
Teacher: Okay, so the equation for the points on a circle is $$x = R \cos(\theta) \qquad y = R \sin(\theta)$$
 can you write $dx$ in terms of $d\theta$?
Student:  I think you'd say $d(R \cos(\theta)$ - right?
Teacher: Yeah, so assume $R$ is constant...
Student: then would you get $-\sin(\theta)$?
Teacher: Close. What is $dx / d\theta$? Have you done derivatives of sin and cos?
Student: derivative of sin is cos, derivative of cos is -sin, derivative of -sin is -cos, derivative of -cos is sin (I think)
Teacher: Right, yeah. 
Student: i'm kind of confused on what I'm doing with $dx/d\theta$, i gues
Teacher: Well, $x = R \cos(\theta)$, so $dx/d\theta = \cdots$. You already said it.
Student: $-\sin(\theta)$?
Teacher: yeah, so what's $dx$?
Student: the same thing?
Teacher: Nope, $dx = - R \sin(\theta) d\theta$
We're not being careful about what $dx$ means by itself here, but let's just go with it.
The meaning is intuitively clear, I think.
ok so what's $dy$?
Student: $-R\cos(\theta)$?
Teacher: Where'd the minus come from, and where's the $d\theta$?
Student: I don't really know.
Teacher: $dy = R \cos(\theta) d\theta$
Does that make sense?
$y = R \sin(\theta) \rightarrow dy/d\theta = R \cos(\theta) \rightarrow dy = R \cos(\theta) d\theta$
Student: hmm, I think so
it makes sense, but I'm not sure, given another problem, that I could do it
Teacher: Which bit is confusing?
Student: what exactly the $dy/d\theta$ means, exactly, I suppose - which equation are we finding the derivative of?
Teacher: We have two equations: $$x = R \cos(\theta) \qquad y = R \sin(\theta)$$
Those give you the x and y coordinates of the points on a circle, as you vary the angle $\theta$.
$dx/d\theta$ tells you how $x$ changes as we change the angle $\theta$.
Student: and then $dy/d\theta$ - how $y$ changes as we change $\theta$?
Teacher: Yeah. As you change the angle going around the circle, both x and y change!
Student: but wouldn't $R$, being a constant, vanish from the derivative?
Teacher: Nope. If I have $y = a x$ then $dy/dx = a$.
Student: Oh, that makes sense, sorry.
Teacher: No problem. So we have $$dx = - R \sin(\theta) d\theta \qquad dy = R \cos(\theta) d\theta$$
Student: okay, i think i follow now
Teacher: Cool. We also know, as you said, that if we move a bit in x and y, the total distance traveled is $$ds = \sqrt{dx^2 + dy^2}$$
Student: right
Teacher: So! Plug in and see what you get 
Student: $ds = \sqrt{(-R\sin(\theta)d\theta)^2+(R\cos(\theta)d\theta)^2}$ - convoluted madness
simplifying
$\sqrt{(-R\sin (x)dx)(-R\sin(x)dx)+(R\cos(x)dx)(R\cos(x)dx)}$
Teacher: Alright, but you can simplify that a lot. Multiply stuff together...
Student: um, let's see, $R^2 \sin^2(x)dx$ for the first part?
and then plus $R^2\cos^2(x)dx$, right?
Teacher: Actually wait a second... $dx = -R \sin(\theta) d\theta$
You're plugging $dx$ and $dy$ into the $ds$ formula...
Student: Let me try this again.
Teacher: what's $dx^2 = (-R \sin(\theta) d\theta)^2$? $= R^2 \sin(\theta)^2 d\theta^2$
So do the same for $dy^2$.
Student: so it'd be $\sqrt{R^2\sin(\theta)^2+R^2\cos(\theta)^2d\theta^2}$
Teacher: YES!
Now factor out the stuff you can factor out...
Oh, and you forgot the $d\theta^2$ in the first term.
Student: um, let's see, $R^2$, and $d\theta^2$
so then we're left with $Rd\theta\sqrt{\sin(\theta)^2+\cos(\theta)^2}$?
Teacher: no. Forgot to square-root the stuff you pulled from the square root 
What's $\sin^2 + \cos^2$?
Student: oh duh, 1, Pythagorean identity
so $Rd\theta$
because sqrt 1 = 1
and 1*anything = anything
Teacher: Yes.
So you have $ds = R d\theta$.
This is a very nice result: the length of an arc of a circle is equal to the radius times the angle of that arc.
This is, actually, how angle is usually defined, but here we derived it from the starting point that a circle is defined by $x = R \cos(\theta)$ and $y = R \sin(\theta)$.
Now, how much arc length do you get if the angle goes from 0 to $2\pi$?
Student: not sure...$2\pi$?
Teacher: Well, let's just do it. $$s_\text{total} = \int_{\theta=0}^{2\pi} R d\theta$$
$$=R \int_{\theta=0}^{2\pi} d \theta = ?$$
Student: $2\pi R$
the circumference of the circle!
which makes sense because $2\pi$ is a full revolution around!
Teacher: Right! You actually just did some multivariable calculus.
Remember we started with the Pythagorean thingy $ds = \sqrt{dx^2 + dy^2}$.
That tells you the length of a bit of curve in the 2D plane.
Then you used a parametrization of the circle, i.e. $x = R \cos(\theta)$ and $y = R \sin(\theta)$ to figure out $dx$ and $dy$ in terms of a single variable $\theta$, and then you did the integral to get the total path length.
Student: oh...wow, that's cool!
Teacher: Yes, it is!
