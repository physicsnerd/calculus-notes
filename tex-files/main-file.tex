\documentclass{book}
\usepackage{graphicx}
\usepackage{amsmath}
\usepackage{float}
\usepackage[english]{babel}
\usepackage[english = american]{csquotes}
\MakeOuterQuote{"}
\usepackage[pdfpagemode=useNone,pdfstartview=FitH,colorlinks=true,linkcolor=blue,citecolor=blue,urlcolor=blue]{hyperref}
\usepackage[all]{hypcap}
\usepackage{emptypage}

\title{Single-Variable Calculus}
\author{physicsnerd}
\begin{document}

\maketitle
\frontmatter
\tableofcontents
\chapter{Introduction}
Calculus is the study of change. 
There are two main branches to calculus: differential and integral calculus. 
Differential calculus studies the slopes of lines, or the rate of change, using the derivative. 
Integral calculus studies the area under a curve. 
Many problems in many fields of science boil down to one of these two problems. 
Here, we will go over how to solve problems in the main fields of calculus, some applications of calculus, and finally some next steps in your study of calculus.
This book (booklet might be more appropriate) does not assume significant background knowledge - while it would be nice for the reader to have an understanding of algebra, and perhaps to have even been through trigonometry in their studies, this book requires none but the most basic algebra - factoring quadratics, or simplifying radicals, or other more advanced algebra topics are explained as they come up in the text. 
Solutions to all problems are found in Appendix II, as this book is intended for both self-study and study with others, in whatever setting that might entail.
If you do have any questions, many resources can be found on the internet, or you can create an issue at this book's github page: https://github.com/physicsnerd/calculus-notes.

\mainmatter
\chapter{Terms to Know}

\input{termstoknow.tex}

\part{Limits}

\subsection{Introduction}

When we talked about derivatives, we relied on the idea of limits - something that could "zoom in" to a point that was not necessarily defined - but we left the actual mechanics of them rather sketchily laid out.

Let's think about what we want a limit to do. Imagine we have the function

\begin{equation*}
    f(x) = \frac{(3x+1)(x-1)}{(x-1)}
\end{equation*}

When we graph it, we get a hole at the point where $x=1$ - the function is undefined there. But imagine if we look at the points really close to that hole on either side - where $x=.99$ or $x = 1.01$. What do those values approach?

\begin{center}
    \begin{tabular}{c|c|c|c}
       $x < 1$  &  $y$ & $x > 1$ & $y$\\
       \hline
        $0.99$ & $3.97$ & $1.01$&$4.03$ \\
        $.999$ & $3.997$ & $1.001$&$4.003$ \\
        $.9999$ & $3.9997$ & $1.0001$& $4.0003$
    \end{tabular}
\end{center}

We want our definition of a limit, when given this function and told to approach $1$, to produce an outcome of $4$.

We also want our limit contraption to spit out the right thing if there {\it isn't} a hole: if we find the limit as $x$ tends to $1$ of $y=x$, our limit should result in $1$. In other words, our definition of a limit shouldn't do something crazy when normal inputs, else it isn't the definition we want.

\subsection{Formal definition}

Now that we know what our definition of limits needs to entail

\subsection{Solving limits}
\subsubsection{Using the definition}



\subsubsection{Proof-free methods}

One does not need to use an epsilon-delta proof every time. Much like with the derivative, you can use the full definition, or you can just use the rules that pop out of the definition (or in this case, tricks for rearranging things) to solve the problem more easily.

The first type of limit you could call an "easy" limit. 
All you have to do is plug in the number you are approaching for your variable, because the function doesn't violate any rules like we described above. 
So for example, if you have the limit

\begin{equation*}
    \lim\limits_{x\rightarrow 1} x
\end{equation*}

You would just get 1. 
Other limits require a little more work. For instance, say we have
\begin{equation*}
    \lim\limits_{x\rightarrow -3}\frac{x^2-x-12}{x+3}
\end{equation*}
We can clearly see that if we just plug it in, there will be a zero in the denominator. 
Instead, we can try messing around with the expression using tools like factoring or rationalizing the denominator. 
In this particular case, the expression in the numerator is very easy to factor:
\begin{equation*}
    \lim\limits_{x\rightarrow -3}\frac{(x+3)(x-4)}{x+3}
\end{equation*}
This then leaves us with
\begin{equation*}
    \lim\limits_{x\rightarrow -3}x-4
\end{equation*}
The limit is now one of our "easy" limits, so we can simply plug in $-3$. The answer is $-7$.

Now we have our final type of limit, the "$\frac{\text{not zero}}{0}$" limit. 
This type of limit is a bit different from the other two types of limits. As an example, let's say we are trying to solve the problem
\begin{equation*}
    \lim\limits_{x\rightarrow 0}\frac{1}{x}
\end{equation*}
To solve this, we must first solve
\begin{equation*}
    \lim\limits_{x\rightarrow 0^+}\frac{1}{x}
\end{equation*}
and
\begin{equation*}
    \lim\limits_{x\rightarrow 0^-}\frac{1}{x}
\end{equation*}
Looking at the second problem first, the $+$ sign means it is a right limit - that is, we are approaching zero from the left. Let's say we have a number line like the one below.

\begin{figure}[H]
\centering
\includegraphics[scale=0.5]{numberline.jpg}
\end{figure}

To approach from the right means that $x$ will be represented by tiny positive numbers, like $0.00001$ and $0.0000001$, that get closer and closer to $0$. 

Thus, the answer to this particular limit will be whatever $\frac{1}{\text{really small positive number}}$ is. 
For instance, if we have $\frac{1}{0.00001}$ we get $100000$; if we have something closer to zero, like $\frac{1}{0.0000001}$ we get $10000000$. 
In other words, we are approaching infinity, so $+\infty$ is our solution.

If we look at the first problem, this time, we are approaching zero from the left, so we are dividing by negative numbers. (It's very easy to check your work for most limits on a calculator by plugging in numbers and seeing what it approaches.) It turns out that this approaches $-\infty$. 

Now, to solve the original limit. 
If we take the two limits we split the original into and their answer is the same, the first limit has a solution, but in our case the second two limits had different solutions, meaning there is no answer. What we're doing here is seeing if there's a big jump between the left and the right or not.

\subsubsection{L'H\^{o}pital's Rule}

Sometimes we get limits that we cannot manipulate, and when you plug in the limit it turns into something nasty and unworkable like $\frac{0}{0}$ or $\pm\frac{\infty}{\infty}$.

You'll notice in the past that when we've taken the derivative of functions they tend to become less disgusting - powers get reduced, constants get thrown out, etc. Perhaps you see where this is going.

talk about geometric approach? slopes match up - but only in certain cases, why only in certain cases

\chapter{Problems}
\begin{enumerate}
    \item $$\lim\limits_{x\rightarrow 3} 2x+5$$
    \item $$\lim\limits_{x\rightarrow 4} \frac{x^2-16}{x-4}$$
    \item $$\lim\limits_{x\rightarrow 9} \frac{\sqrt{x}-3}{x-9}$$
    \item $$\lim\limits_{x\rightarrow 0} \frac{1}{x^2}$$
    \item $$\lim\limits_{x\rightarrow 1} \frac{\sqrt{x}-1}{x-1}$$
\end{enumerate}

\part{Integral}

\chapter{Intuition Behind the Integral}
The point of the integral is to find the area under a curve. 
One of the interesting applications of integration is finding the displacement of an object - the displacement corresponds to an object's trajectory; that is, the area under a graph of velocity versus time is displacement.

While we don't actually need to do the following, this is generally what is happening. 

\begin{centering}
\begin{figure}[H]
\caption{Riemann Sum}
\includegraphics[scale=0.8]{../rieman.jpg}
\end{figure}
\end{centering}

Basically, we pick a specified width, let's say $w$. 
Then, we create several rectangles, with a width $w$ and a height that is just under the curve we are trying to find the area of. 
Then, we do this again, but this time the height is just above the curve. 
We take the average of the two numbers, and get the area of the curve. 
The smaller we make $w$, the more accurate the number, and at width $0$, we basically have lines, which fit the curve perfectly. 
This is the point of an integral.
\chapter{Basic Integration Rules}
Below is a table of some of the basic outputs of integrals. Some of these will make more sense after going through derivatives.

\begin{tabular}{l|l}
    $f(x)$ & $\int f(x) \, dx$\\
    \hline
     $0$ & $0+c$ \\
     $1$ & $x+c$ \\
     $D$ & $Dx+c$ \\
     $2x$ & $x^2 + c$ \\
     $x$ & $x^2/2 + c$ \\
     $x^2$ & $x^3/3 + c$ \\
     $x^n$ & $x^{n+1}/n+1 + c$
\end{tabular}

In these, $c$ and $D$ are constants and $x$ is some variable.

Note that after every integral, we must put $dx$ or $dt$ or $d$ and then whatever variable we are using in our integral. 
The second important thing to note is that these results have $c$ after them because they are ``indefinite'' integrals - that is, because we don't have a specific bound on the integral, we must add a constant $c$ to show that there is some range in what the answers will be. 
This is again something that will make more sense after going over derivatives.

So there are indefinite integrals, but there are also "definite" integrals. 
We write them as $\int^b_a$. Basically, let's say we have the integral $\int x \, dx$. 
We integrate the inside according to the table, getting $x^2/2$, but instead of adding the constant $c$, we then plug in $b$ and $a$ in for $x$ as follows: 
\begin{equation*}
    (b^2/2) - (a^2/2)
\end{equation*}

This gives our answer, and we do not need to put in $c$.

There is another important thing about integrals. For an integral such as $\int x+x \, dx$, we can change this into

\begin{equation*}
    \int x \, dx + \int x \, dx
\end{equation*}

We can also factor out constants; for example, given $\int 2x + 3x \, dx$, we can change this into 

\begin{equation*}
    2\cdot \int x \, dx + 3 \cdot \int x \, dx
\end{equation*}

Also remember that you can simplify within a problem; i.e., given $x(x^2 + x^3)$ you can then multiply and get $x^3 + x^4$. 
Finally, there is a helpful rule for integrals - the product rule. 
If you have a function $v(x)$ and a function $u(x)$, and you wish to find the derivative of $u(x)\cdot v(x)$, you can follow the rule $u\cdot v - \int v\frac{du}{dx}\, dx$.
\chapter{Derivatives and Integrals}
Derivatives and integrals are inverses of each other. 
That is, they undo each other, like division undoes multiplication and subtraction undoes addition. 
(This is called the "fundamental theorem of calculus".)
So, because when going from an integral back to a derivative there is some uncertainty which graph you are looking at (parallel lines, for instance, have the same slope, but they have different "heights"), you have to add the $c$ when solving an indefinite integral. 
\chapter{Formal Definition of an Integral}
\section{Summations}
A summation is a simple method of notation that shortens long repetitive addition operations. Let's look at an example:

\begin{equation}
\sum\limits_{n=0}^{n=5} 2n
\end{equation}

Here, the expression $n=0$ under the summation symbol means that to start the sum, we plug in $0$ for $n$. The expression $n=5$ on top means we continue until we have plugged in $5$ for $n$. Expanded, this summation means

\begin{equation}
2\times 0+2\times 1+2\times 2+2\times 3+2\times 4+2\times 5
\end{equation}

which means the answer is $30$. Think back to the section on the intuition behind integrals for why we need this. We're really summing up a bunch of small rectangles. 
\section{Formal definition of indefinite integrals}
\section{Formal definition of definite integrals}

\chapter{Problems}
Solve the following integrals:

\begin{enumerate}
    \item $\int x \, dx$
    \item $\int x^2-5 \, dx$
    \item $\int^5_0 x^2 \, dx$
    \item $\int^5_3 x^2 \, dx$
    \item $\int^5_3 x^2 + 1 \, dx$
\end{enumerate}

\part{Differential}

\chapter{Intuition Behind Derivatives}
Differential calculus is all about finding the slope of a curve. 
First, let's consider a normal case, finding the slope of a straight line. 
We take in two points, and find the change in x and change in y ($\Delta x$ and $\Delta y$). 
This can be written as $\frac{\Delta y}{\Delta x}$ or $\frac{\text{rise}}{\text{run}}$.

\begin{figure}[H]
\caption{Average Slope}
\includegraphics[scale=1]{../download.png}
\end{figure}

The above diagram illustrates this process on a curve. 
We pick two points on the line and find the average slope. 
However, this process is clearly not very accurate. 
Imagine that we move the two points closer and closer together. 
The closer they are, the more accurate the slope is for that small section of line, until eventually we make it so that there is only one point. 
That is, we are finding instantaneous slope, or the slope at one point of the line.

The derivative does that for the whole line, taking in a function and giving a new function that represents the slope. 
For example, let's say we shoot a cannonball.

\begin{figure}[H]
\caption{Trajectory of cannonball}
\includegraphics[scale=1]{../imgres.png}
\end{figure}

Here, the y-axis represents the height of the cannonball in meters and the x-axis indicates time in seconds. 
Notice that at the beginning the slope is positive, until eventually the graph reaches its peak and the slope is zero, and then the graph starts going down and the slope becomes negative. 
The derivative of this graph would then look something like this:

\begin{figure}[H]
\caption{Derivative of trajectory of cannonball}
\includegraphics[scale=0.8]{../derivative.png}
\end{figure}

This is what is happening "behind the scenes" when we calculate derivatives.
However, it is important to note that, going back to the example of the cannonball, where the cannonball started does not affect the slope. 
If the whole graph is shifted up or down, the slope stays the same (as long as the proportions are kept the same). 
This, it turns out, is why we need to add the $c$ in a indefinite integral. 
\chapter{Derivatives and Integrals}
Derivatives and integrals are inverses of each other. 
That is, they undo each other, like division undoes multiplication and subtraction undoes addition. 
(This is called the "fundamental theorem of calculus".)
So, because when going from an integral back to a derivative there is some uncertainty which graph you are looking at (parallel lines, for instance, have the same slope, but they have different "heights"), you have to add the $c$ when solving an indefinite integral.
\chapter{Basic Derivative Rules}
Below is a table with some solutions to various derivatives. Note that $c$ is a constant and $n$ is non-zero. Note that the $'$ symbol is pronounced prime, and is another way to write that we are taking the derivative. 


\begin{tabular}{c|c}
    $f(x)$ & $\frac{df}{dx} = f'$\\
    \hline
       $7$  & $0$ \\
        $c$ & $0$ \\
        $x$ & $1$ \\
        $cx$ & $c$ \\
        $x^2$ & $2x$ \\
        $cx^2$ & $2cx$ \\
        $x^3$ & $3x^2$ \\
        $x^n$ & $nx^{n-1}$ \\
        $\sin x$ & $\cos x$ \\
        $\cos x$ & $-\sin x$ \\
        $- \sin x$ & $- \cos x$ \\
        $-\cos x$ & $\sin x$
\end{tabular}

Like integrals, derivatives are distributive over addition.
There are some interesting rules that make working with derivatives easier: the product rule, the quotient rule, and the chain rule. 

Let's start with the chain rule. 
If we have a function within a function in a derivative, like $\sin(\cos(x))$, we can assign a name to each function, like $g(x) = \cos(x)$ and $h(x) = \sin(x)$, and then follow this simple rule: $h'(g(x)) \cdot g'(x)$. 
So what does this mean? 
Well, let's plug it in. 
Plugging this in gives $\sin'(\cos(x)) \cdot \cos'(x)$. 
So the derivative of $\sin$ is $\cos$, so we write $\cos(\cos(x))$ for the first part, and the derivative of $\cos$ is $-\sin$, so we write $-\sin(x)$ for the second part. 
So now we have $\cos(\cos(x)) \cdot -\sin(x)$. 
This is the derivative of our original function.

Now, let's examine the product rule. 
If we have two functions, $u(x)$ and $v(x)$, and we want to take the derivative of $u\cdot v$, then we can do $u \cdot \frac{dv}{dx}+ v\cdot\frac{du}{dx}$. 
In other words, we multiply the first function by the derivative of the second function and then add the second function times the derivative of the first. 
For example, if we have $\sin(x)\cdot\cos(x)$ we would do $u(x) = \sin(x)$ and $v(x) = \cos(x)$. 
Then we would plug it in to our formula: $\sin(x)\cdot\frac{d}{dx}\cos(x) + \cos(x)\cdot\frac{d}{dx}\sin(x)$ which simplifies to $\sin\cdot -\sin(x) + \cos(x)\cdot\cos(x)$ or $-\sin^2(x)+\cos^2(x)$, which is our derivative of the original function.

Finally, the quotient rule. 
If we have two functions, $g(x)$ and $h(x)$, and we wish to take the derivative of $\frac{g(x)}{h(x)}$, we can follow the rule $\frac{g'(x)h(x)-h'(x)g(x)}{[h(x)]^2}$. 
While this may look rather complicated, it really isn't too bad. 
Let's use the example of $g(x) = \sin(x)$ and $h(x) = \cos(x)$. 
We plug it all in to get $\frac{\sin'(x)\cdot\cos(x) -\cos'(x)\cdot\sin(x)}{\cos^2(x)}$
which simplifies to $\frac{\cos(x)\cdot\cos(x)--\sin(x)\cdot\sin(x)}{\cos^2(x)} = \frac{\cos^2(x)+sin^2(x)}{\cos^2(x)} = \sin^2(x)$. 
So therefore, $\sin^2(x)$ is our derivative.
\chapter{Formal Definition of a Derivative}
Now, let's say you forget these rules, or come across something you can't use these rules on. In this case, you can use the formal definition of a derivative. When looking at this definition, remember that $\Delta$ means "change in" whatever variable comes afterwards (i.e., $\Delta x$ means the change in x). Given a function $f(t)$, the derivative of $f(t)$ ($\frac{df(t)}{dt}$) is \begin{equation}
    \lim\limits_{\Delta t\rightarrow 0}\frac{\Delta f}{\Delta t} = \lim\limits_{\Delta t\rightarrow 0}\frac{f(t+\Delta t)-f(t)}{\Delta t}
\end{equation}
Let's try using this on a derivative - say, the derivative of $f(t)$ where $f(t) = t^2$. We have to solve $\lim\limits_{\Delta t\rightarrow 0}\frac{f(t+\Delta t)-f(t)}{\Delta t}$. Remember that the rule is that the input to $f$ must be squared, so for the first term, we need to square $t+\Delta t$ - $(t+\Delta t)(t+\Delta t) = t^2+t\Delta t+t\Delta t + \Delta t^2 = t^2+2t\Delta t+\Delta t^2$. That's the first term on top of the fraction. The second term is just $f(t)$ - which we've been told equals $t^2$. Since we subtract this from the first term, the two cancel, leaving $\frac{2t\Delta t+\Delta t^2}{\Delta t}$. We then divide, getting $2t+\Delta t$. Now we have to apply the limit in front. Remember that if there will be no mathematically impossible results, we can just plug in the limit - i.e., in this case, $2t+0$ or just $2t$, which is our result. We can of course check this using our earlier exponent rule (which we can actually derive; see exercise six and solution), and $2t$ is still the result.


\chapter{Problems}
\begin{enumerate}
    \item $5x$
    \item $6x^3 - 9x + 4$
    \item $2t^4 - 13t$
    \item $x^{-1}$
    \item $\sqrt{x}$
    \item Derive the exponent rule using the formal definition of a derivative (the rule being that if you have a function $t^n$, the derivative will be $(nt^{n-1}$).
\end{enumerate}

\part{Applications}

\input{applications.tex}
\backmatter
\part{Appendices}
\appendix
\chapter{Circles with Calculus}
\section*{Formal Proof}
It is given that there is a circle, with a radius $r$, and the circumference of a circle is $2\pi r$. If we think of the circle as a series of concentric rings, we can then integrate $2\pi r$ over the interval $0$ to $r$, or from the center of the circle to the outer rim of the circle. So we now have the integral $\int^r_0 2\pi r\, dr$. We then may simplify by the definition of an integral to $\frac{2\pi r^2}{2}\mid^r_0$, and simplify yet further to $\frac{2\pi r^2}{2}-\frac{2\pi 0^2}{2}$, which then becomes $\frac{2\pi r^2}{2}$ or $\pi r^2$. Therefore, we have shown that the area of a circle with radius $r$ is $\pi r^2$.
\section*{Intuition}
\input{appendix1.tex}
\chapter{Worked Solutions}

\input{solutions.tex}
\end{document}
