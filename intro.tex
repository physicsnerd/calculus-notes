Quite a few people have told me they dislike calculus, or find it a pain. This is, to some extent, understandable. I know very few people who enjoy memorizing formulas, and a common refrain in math is "why not just use [insert calculator of choice here]?" 

Math does have very broad applications, of course, and calculus is no exception - not just in the sciences, but in more humanities oriented subjects as well (though I do admit you will likely not need calculus analyzing literature...then again, you never know!), which are often slightly less recognized. I try to give some examples of these as well.

However, really I think the problem is that when it seems like you're just throwing numbers (or letters, as the case may be) at the wall in an attempt to get them to stick, math seems incredibly frustrating and pointless. Math does have a lot of intuitive reasoning, but it's often not mentioned at all.

There's a few examples of this that can be hit upon quite quickly in algebra. Take finding the area of something, a concept \textit{integral} to calculus (sorry). We're told that for a rectangle it's height times width, and then it's left at that.

But let's actually think about this! Let's pretend we did not go through geometry and try to figure out what the formula for area is - we might be able to guess intuitively it has something to do with length and width, because we know the size of the 2d shape we're thinking about increases as those things increase...but what exactly do we do with length and width?

Well a good first step is to start by defining properties we know the formula for area should have, based off of what we intuitively know for area. For instance, say we have shape one, with a certain length and width. If we place a copy of that shape next to itself - doubling the width - the area overall should double, just from an intuitive sense. So we know that $A(l, 2w) = 2 A(l, w)$. We know this is true if we stack the copies too, and that it's true if we stack three copies, or four, or however many we'd like. So we know that $A(x l, w) = x A(l, w)$ and $A(l, xw) = x A(l,w)$. Alright, we've defined what must be true about area based on some intuitive properties...the next step is generally doing some manipulating.

In this case, what those two rules told us is that we can pull numbers out from inside of the area formula. For instance, $l = 1\cdot l$, and $w = 1\cdot w$...so we can rewrite our formula as $A(l,w) = lwA(1,1)$. What's $A(1,1)$? It's effectively the idea of units. Are we measuring area in light-years? Nanometers? $A(1,1)$ needs to be the area of a single square [unit].

The third bit generally has to do with picking our final equation to make it pretty. We could say $A(1,1) = 13$. There is nothing that stops us from doing this. Area would still follow those intuitive properties we mentioned. \textit{We choose that $A(1,1) = 1$ because it's convenient.} (See the resources section - there's a very good book from which this example is lifted and goes into these ideas in more depth.)

All of the intuition there, the reason why we talk about length times width, and why the equations are the way they are, is not mentioned. And so on, throughout calculus - proofs are mentioned as an aside  if at all (and often when actually mentioned they're gory cramped affairs that explain very little), applications are whole new things to learn instead of consequences of the base ideas, and just in general it feels like a slog.

And here is the point where I'm supposed to say 'but no longer! For just \$19.99 you too can have the shiny Calculus 9000!' This is disingenuous at best. Different explanations work for different people. The lovely thing about calculus, having been around for hundreds of years and being a relatively commonly taught subject, is that there are many, many resources out there, many of which are free (I list some that I've found particularly nice at the end of this). If the explanations here don't make sense to you...try some others! The repetition of the ideas from a new angle will likely assist. 

If you've tried a bunch of different resources and something just desperately does not make sense - ask people who you know are alright at explaining things to try to explain it to you. Find a forum you can ask on, like r/learnmath or math.stackexchange.com. Ask a teacher - if you don't follow your math teacher's explanation, try a science teacher. Try applying the idea to a few problems and see what happens. Talk to a rubber duck (or a stuffed animal, a long-suffering friend, a convenient squirrel, etc) as you work through the problems, explaining what you know and why you are approaching each step the way you are. There is no shame in something taking some time to click.

Quite frequently it is not your fault something doesn't make sense, and the solution is not banging your head against a brick wall. Find a different wall that perhaps has a door.

Finally, calculus is often made harder by the gross algebra or arithmetic in it. (I say this as someone who has many times screwed up a derivative because I forgot, for instance, that 4+2 is not 8.) If you don't remember some of the algebra and trig, go back and look up videos on it, like on Khan Academy or from PatrickJMT, or grab the \textit{For Dummies Algebra Workbook} (the workbook has a bunch of worked through problems, which I found helpful back when I was reviewing algebra). 